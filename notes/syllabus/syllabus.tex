%========================================================================================
% Compilation should work with PDFLaTeX
%========================================================================================
% Type of document and general formatting
\documentclass[a4paper,12pt]{article}

\usepackage[left=2.25cm,right=2.25cm,top=2.25cm,bottom=2.25cm]{geometry}
%\linespread{1.25}

%========================================================================================
% These packages are for language and font settings
\usepackage[english,spanish,activeacute]{babel}   % Languages
\usepackage{mathpazo}					          % Math font
\usepackage[utf8]{inputenc}				          % Special symbols
\usepackage[T1]{fontenc}				          % T1 Encoding of font
\usepackage{lmodern}
\usepackage[mono=false]{libertine}					          % Text font

%========================================================================================
\usepackage{amsmath,amsfonts,amssymb}	% Math symbols
\usepackage{dsfont}						% Math symbols like R for reals...

%========================================================================================
% Bibliography packages
\usepackage{natbib}
\usepackage{bibentry}
\bibliographystyle{apaurl}

%========================================================================================
% Other packages
\usepackage{graphicx}
\usepackage{longtable}
\usepackage[svgnames]{xcolor}

%========================================================================================
\usepackage{accents}
\newcommand*{\dt}[1]{%
	\accentset{\mbox{\large\bfseries .}}{#1}} % Larger dot for time derivative


\usepackage{hyperref}
\hypersetup
{
    pdfauthor={Rafael Serrano-Quintero},
    pdfsubject={Economic Growth},
    colorlinks = {true},
    linkcolor = {FireBrick},
    citecolor = {FireBrick},
    urlcolor = {RoyalBlue},
}

\usepackage{appendix}
\usepackage{marvosym}
\usepackage{enumitem} %For enumerating with letters with option [a)]
\usepackage{fancyvrb}  %To reduce font size in verbatim environment
\usepackage{epstopdf}
\usepackage[flushleft]{threeparttable}
\usepackage{pdflscape}
\usepackage{subcaption}
\usepackage{booktabs}
\usepackage[super]{nth}
\usepackage{float}
\usepackage{dirtree}

\renewcommand{\textbullet}{\ensuremath{\bullet}}
\newcommand{\source}[1]{\caption*{\tiny Source: {#1}} }

\newtheorem{definition}{Definition}
\newtheorem{theorem}{Theorem}
\newtheorem{remark}{Remark}

%========================================================================================
					% === Title, thanks, and author data === %
%========================================================================================

\begin{document}
\nobibliography{../matlab_intro}

\title{\textbf{Introduction to Matlab}}
\author{Rafael Serrano Quintero \\
University of Barcelona}
\date{\nth{2} Semester 2021}

\maketitle

\section{Information}

\begin{itemize}
    \item Instructor: Rafael Serrano Quintero
    \item Office: 416 (Diagonal 696 Building)
    \item Email: \href{mailto:rafael.serrano@ub.edu}{rafael.serrano@ub.edu}
    \item Office Hours: Mondays 09:00 - 11:00 (confirm by email 24 hours before)
\end{itemize}

\section{Course Objectives}

The goal of the second part of the course \textit{``Introduction to Matlab and Stata''} is to learn the basics of Matlab. Students will then be able to apply the acquired knowledge to problem solving in their research projects.

\section{Grading Policy}

\begin{itemize}
    \item \underline{Problem Sets} ($30\%$ of the final grade).
    Students will get two problem sets to solve at home and some exercises that must be solved and explained in class. Answers must be submitted formatted in { \LaTeX } with the original code as well. The students will send me one compressed file named \verb+SURNAME_NAME_PSX.zip+ where \verb+X+ is the number of the problem set. The file must have the following structure:
    
    \begin{minipage}[c]{0.8\textwidth}
        \dirtree{%
            .1 SURNAME\_NAME\_PSX.zip.
            .2 tex.
            .3 SURNAME\_NAME\_PSX.tex.
            .3 SURNAME\_NAME\_PSX.pdf.
            .2 code.
            .3 \ldots{} \begin{minipage}[t]{5cm}
                All Matlab codes{.}
            \end{minipage}.
        }
    \end{minipage}
    \item \underline{In-class Solutions} ($10\%$ of the final grade). Students will explain in class how to solve one of the exercises and show their code, explain their train of thought, and answer questions from their classmates.
    \item \underline{Exam} ($60\%$ of the final grade). Students will have 24 hours to complete a take-home exam with exercises similar to those of the problem sets.
\end{itemize}

\section{Topics and Organization}

\begin{enumerate}[label={\textbf{Session \arabic*}}]
    \item Matlab preliminaries. 
    \begin{itemize}
        \item First interactions. Script vs Command Window.
        \item Creating Variables. Basic Operations. Arrays and Matrices.
        \item Control Flow. Plots. Functions.
    \end{itemize}
    \item Importing and manipulating data. Polynomial fit and evaluation. Nonlinear least squares.
    \item Basics of root finding, numerical differentiation and integration.
    \item Basics of numerical optimization.
\end{enumerate}

\section{Materials}

I will send you slides, problem sets, and codes that we will use during class but here are other materials from which I have taken a lot.

\begin{itemize}
    \item \href{https://mitpress.mit.edu/books/numerical-methods-economics}{\bibentry{judd1998numerical}}
    \item \href{https://mitpress.mit.edu/books/algorithms-optimization}{\bibentry{kochenderfer2019optimization}}
    \item \href{https://people.lu.usi.ch/gruberp/MatlabMasterScript.pdf}{Peter H. Gruber --- Script Solving Economics and Finance Problems with MATLAB}
    \item \href{https://cheatsheets.quantecon.org/index.html}{QuantEcon Cheatsheet} ---  for Matlab, Python, and Julia.
    \item \href{https://quantecon.org/lectures/}{QuantEcon Lectures.} These are written for Python and Julia but many ideas port to Matlab easily.
    \item \href{https://www.sas.upenn.edu/~jesusfv/teaching.html}{Jes\'us Fern\'andez-Villaverde --- Computational Methods for Economists}
    \item \href{http://www.numerical-tours.com/matlab/}{Numerical Tours --- Gabriel Peyr\'e}
\end{itemize}

\section{Calendar}

\begin{table}[htbp]
\centering
\begin{tabular}{@{}lccc@{}}
    \toprule
    \multicolumn{1}{r}{} & Session 1 & Session 2 & At Home \\ \midrule
    01-10-2021           & Matlab Preliminaries  & Data Fitting & Solve PS1  \\
    15-10-2021           & \begin{tabular}[c]{@{}c@{}}Root Finding, Numerical \\ Differentiation and Integration\end{tabular} & Grade PS1    & PS2 \\
    22-10-2021           & Optimization & Grade PS2 \\ \bottomrule
\end{tabular}
\end{table}


\end{document}